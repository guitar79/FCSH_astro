\section{결과 및 토의}

\subsection{천체망원경 모터 포커서 컨트롤러 구동 시스템 개발}

\subsubsection{온습도 센서를 이용한 온도 및 습도의 측정}

이 연구를 진행하는 데 아두이노를 사용하는 것이 가장 기본이라고 판단하였기 때문에 아두이노로 실행할 수 있는 것들 중 쉬운 축이라고 생각되는 온습도 센서(dht22)를 활용하여 온습도를 측정하는 일이었다. 그림6과 같이 기판을 짜고 코드를 입력하면 serial 모니터에 온도와 습도가 delay만큼의 간격을 두고 계속 출력된다.
이를 응용하여 그림 7처럼 oled(oled1306)에 온도와 습도를 실시간으로 출력하는 프로그램을 만들 수도 있다.

\subsubsection{스테핑모터의 회전}

스테핑 모터의 종류는 2가지가 있다. 하나는 전선이 6개가 연결된 것과 전선이 4개가 연결되어 있는 것이다. 우리가 찾은 코드는 전선이 4개만 연결하는 것에 대한
▲그림 8. DRV8825의 구조
코드였기 때문에, 구멍인 6개인 것을 4개인 것에 대응시켰다. 전선이 6개인 것의 1번, 3번, 4번, 6번을 연결하면 구멍이 4개인 것과 같은 효과를 낼 수 있다. (순서가 반대로 된다면 모터의 회전 방향이 반대가 될 것이다) 또한, 모터으 회전 방향을 제어하기 위해서는 아두이노에 모터드라이버를 사용하여야 한다. 모터드라이버는 drv8825를 사용한다. 스테핑 모터에 있어서 이 모터드라이버가 있으면 움직임을 더욱 정밀하게 설정할 수 있는데, 아두이노의 초기 설정에서 M0, M1, M2(MODE)의 값이 1이냐 0이냐에 따라 풀스텝(1.8)에서부터 1/32스텝까지 한번 실행할 때마다 회전시킬 수 있다. (delay에 따라회전 속도를 조절할 수 있다.)(표1 참조)

\subsection{자동초점조절}

자동초점조절 알고리즘은 모터포커서를 다 제작한 뒤에 구현하여도 문제가 없을 것으로 판단하여 자동초점조절 알고리즘 구현은 모터포커서를 다 제작할 뒤에 연구하도록 할 것이다.
