\section{결론 및 제언}
	
\subsection{결론}

본 연구를 통하여 모터 초점 조절 장치 및 펌웨어 개발, 무선통신 개발, ASCOM 드라이버 개발 등을 통하여 자동 초점 조절 장치 개발을 위한 기반을 마련하였다. 이를 통하여 기존에 있던 초점 조절 장치의 성능을 무선통신 등의 여러 면으로 개선하여 사용자들에게 편의성을 제공하였다. 본 연구가 나아간다면 성능이 좋고 여러 가지 기능이 존재하는 자동 초점 조절 장치를 개발할 수 있을 것이다.

\subsubsection{본 시스템의 장점}
본 연구에서 제작된 ASCOM 드라이버의 가장 큰 특징은 직접 개발한 모터제어기를 사용하였기 때문에 그 연동성이 매우 뛰어나다는 것이다. 예를 들어, 모터제어기에 들어있는 마이크로스텝을 바꾸는 기능은 직접 제작한 ASCOM 드라이버에 그대로 같은 기능을 넣을 수 있었다. 또한 펌웨어에는 없는 기능이더라도 ASCOM 드라이버를 활용하여 더 좋은 활용 방안들을 찾아낼 수 있으며, 망원경의 특성에 따라서 backleash와 같이 필요한 기능들 또한 추가할 수 있을 것이다.  망원경에 따라서 다른 개발 환경을 가질 수 있으므로 연동성이 좋다는 것이 큰 장점으로 작용한다.

또한 본 모터제어기와 드라이버는 ASCOM standard protocol을 사용하였기 때문에 다른 모터제어기와 드라이버사이의 ASCOM을 통한 연결 또한 가능하다. 즉, 모터제어기만 존재할 때에도 다른 드라이버에 연결시켜 ASCOM standard protocol을 사용하여 모터의 초점을 조절할 수 있다. (혹은 그 반대의 경우 또한 가능할 것이다) 이런 경우에는 직접 개발한 드라이버의 여러 기능을 사용할 수 없다는 단점이 존재하겠지만 이 또한 호환성이 좋다는 것이므로 장점으로 작용할 수 있다.

\subsubsection{기존 모터초점조절시스템과의 차이점 및항개선 사항}
직접 개발한 펌웨어에서도 여러 장점을 찾을 수 있다. Micro Touch제품의 경우, 앞서 나열한 것 외에도 스텝의 초기화 기능이 없기 때문에 모터제어기를 끄고 망원경을 컨트롤하는 등의 변수가 생기면 그 전으로 돌아가기 어렵게 된다. 또한 자신이 원하는 스텝의 위치로 바로 옮기는 것 또한 어려운데, 이 드라이버에서는 원하는 position으로 바로 이동하는 기능또한 지원하기 때문에 사용하는데 있어 큰 편리함이 있다. 

본 연구에서 제작한 모터제어기는 기존에 판매되고 있는 제품들의 단점을 보완하기 위해 일정 스텝 이동, 연속적 스텝 이동, 온도 및 습도 표시 등 여러 가지 편리한 기능들을 추가하였으며, 이런 여러 가지 기능들을 효율적으로 사용하기 위해 메뉴 시스템을 제작하여 이런 기능들을 최대한 사용할 수 있도록 하였다. 게다가, 크기가 작고 가볍기 때문에 휴대성에서도 큰 이점을 지니고 있다.

\subsection{제언}
제작한 펌웨어와 ASCOM 드라이버는 개발이 쉽다는 장점이 존재한다. 때문에 펌웨어와 ASCOM 드라이버는 같이 업데이트를 하기 쉬우며, 각 망원경의 특징에 맞게 펌웨어와 ASCOM 드라이버들을 활용하게 되면 각 망원경의 성능을 크게 올릴 수 있는 초점조절 시스템을 개발할 수 있을 것으로 전망된다.
특히나 MFC 라이브러리를 활용한 ASCOM 드라이버의 개발을 완성하게 된다면 효율적인 개발이 가능해지기 때문에 다른 시스템에도 쉽게 적용해, 각 망원경에 최적화된 시스템을 개발할 수 있을 것이다.


\subsection{추후 연구}


\subsubsection{태양, 행성, 달 등의 자동초점조절 알고리즘 개발}

앞서 언급한 것처럼 별은 점 광원이므로 FWHM이나 HFD 등을 이용하여 초점을 정확하게 조절할 수 있으나 태양, 행성, 달 등 점 광원이 아닌 천체는 그 경계선을 이용하여 초점을 판단할 수 있을 것이다. 


\subsubsection{기능 개선}

현재 개발된 GSfocus는 초보 단계로서 개선할 수 있는 부분들이 아직 많이 있다. 

\begin{description}[font=$\bullet$~\normalfont\scshape\color{red!50!black}]
	\item [MCU 변경] STM32L432KC는 ARDUINO NANO보다 성능이 좋은 Arduino로, 방향만 반대일 뿐 핀의 순서와 종류가 모두 같아 ARDUINO NANO에 넣었던 펌웨어를 그대로 사용할 수 있다.
	\item [Heating system 추가] 날씨가 추운 날에는 모터가 얼어서 돌아가지 않거나 렌즈에 서리가 껴서 초점이 맞아도 맞지 않은 것으로 판단할 수 있다. 따라서 이를 예방하기 위하여 열선을 깔아서 DHT22에서 측정한 온도를 바탕으로 특정한 온도 이하로 내려가게 된다면 열선이 활성화될 수 있게 할 수 있다.
	\item [EEPROM 활용] EEPROM은 Arduino 내부에 저장된 비휘발성 메모리로, 컴퓨터의 ‘RAM’과 같은 역할을 하고 있다. 비휘발성이기 때문에 Arduino를 초기화하거나 껐다가 다시 켰더라도 정보를 저장하고 있다. 
	Arduino별로 한 EEPROM의 주소에 들어갈 수 있는 수의 크기가 달라진다. ARDUINO NANO는 4KB의 EEPROM을 지원하므로 0~255까지의 수를 한 번에 저장할 수 있다. 이렇게 저장할 수 있는 수가 작으므로, 여러 가지 주소를 활용하여 큰 수 또한 나타낼 수 있다.(수를 진법으로 바꾸는 과정과 유사함) 실제로 이를 기반으로 펌웨어를 제작하여 보았지만, 수가 약 32000 이상으로 넘어가는 상황에서는 갑자기 수가 이상하게 커지는 오류가 발견되었고, 이를 고쳐야 할 것이다.
	\item [모터 연결 상태 체크 기능 추가] 모터의 연결 상태는 펌웨어를 실행하는 데 아주 중요하다. 만약 펌웨어가 실행되는 도중에 모터가 연결되지 않으면, 스위치를 움직였을 때 스위치의 숫자는 움직이지만, 모터는 움직이지 않아 결과적으로 숫자의 오류를 불러일으킨다. 또한, 모터를 펌웨어가 실행되는 도중에 연결선을 뽑으면 펌웨어에 에러가 일어나는데, 이 경우 다시 모터를 꽂더라도 정상적으로 실행이 되지 않는다. 따라서 이런 여러 상황에 대하여 모터의 연결 상태를 대비한 에러 코드를 설정해야 숫자와 모터가 오차를 일으키는 일이 없을 것이다.
\end{description}
