\section{결론 및 제언}

\subsection{결론}
	
본 연구는 천체 망원경 모터 포커서 컨트롤러인 GS-touch를 개발하는 것이고 개발 및 시험 동작에 성공하였다.  GS-touch는 $12 \textrm{V}$로 구동되며 2상 바이폴라 스테핑 모터가 장착된 모터 포커서를 구동할 수 있으며 본 연구 결과물은 다음과 같이 요약할 수 있다. 

첫째, 아두이노 나노를 기반으로 2상 바이폴라 스테핑 모터 드라이버, OLED 디스플레이어, 온습도 센서 등이 연결된 모터 포커서 컨트롤러인 GS-touch의 하드웨어를 제작하여 구동에 성공하였다. 전원은 DC 12V를 사용하며, 4개의 버튼으로 메뉴를 선택하거나 모터를 정방향 역방향으로 회전시킬 수 있도록 설계하였다.

둘째, GS-touch를 구동할 수 있는 펌웨어을 제작하였다. 펌웨어의 기능은 DRV8825를 이용하여 스테핑 모터를 제어할 수 있도록 하였고, DHT22를 이용하여 온도, 습도 값을 읽어들여 OLED 디스플레이어에 표현할 수 있도록 하였다. 또한 버튼 조작으로 OLED 디스플레이어에 표시되는 메뉴를 조정할 수 있도록 하였다. 

셋째, GS-touch 전용의 ASCOM 드라이버를 개발하여 GS-touch를 PC에 연결하여 ASCOM을 지원하는 MaximDL, FocusMAX 등의 소프트웨어로 구동하는데 성공하였다.

개발된 GS-touch는 기존 제품에 비해 다음과 같은 장점을 가지고 있다. 

첫째, GS-touch는 2상 바이폴라 스텝모터 드라이브인 DRV8825를 사용하여 허용 전류 값이 $3 \textrm{A}$로 micro touch에 비해 월등히 크다. 사진 관측의 경우에는 리듀서, CCD 등 무거운 촬영 장비를 장착하게 되는데 DRV8825의 스펙은 무거운 사진 관측 장비를 장착하고 초점을 조절하는데 무리가 없다. 또한 DRV8825는 1 부터 32 마이크로 스텝까지 펌웨어로 설정 가능하여 포커서의 기어비에 따라 모터 회전 속도를 쉽게 조정할 수 있다. 

둘째, GS-touch는 OLED 디스플레이어를 사용하여 2상 바이폴라 스텝모터 드라이브인 DRV8825본 연구에서 제작된 ASCOM 드라이버의 가장 큰 특징은 직접 개발한 모터제어기를 사용하였기 때문에 그 연동성이 매우 뛰어나다는 것이다. 예를 들어, 모터제어기에 들어있는 마이크로 스텝을 바꾸는 기능은 직접 제작한 ASCOM 드라이버에 그대로 같은 기능을 넣을 수 있었다. 또한 펌웨어에는 없는 기능이더라도 ASCOM 드라이버를 활용하여 더 좋은 활용 방안들을 찾아낼 수 있으며, 

셋째, GS-touch는 DHT22 온습도 센서를 사용하여 온도 뿐 아니라 습도도 파악할 수 있다. 

넷째, GS-touch는 크기가 작아 천체망원경에 장착하기가 용이하다. 

\subsection{제언}

현재 개발된 GS-touch는 초보 단계로서 개선할 수 있는 부분들이 아직 많이 있다. 다음 버전의 하드웨어 부분에서는 버튼 스위치를 좀더 내구성이 좋은 부품으로 교체하는 것이 좋을 것으로 생각된다. 또한 OLED 와 센서의 전원은 아두이노에서 출력되는 전원을 사용하여 외부 전원을 연결하지 않아도 메뉴 조작이 가능하도록 개선할 생각이다. GS-touch 다음 버전 제작시에 개선해야 할 내용을 정리하면 다음과 같다. 

\begin{description}[font=$\bullet$~\normalfont\scshape\color{red!50!black}]
	\item [Backleash 보정 기능 추가] 모터 포커서를 반대 반향으로 구동할 때 발생할 수 있는 backleash 값을 측정하여 보정할 수 수 있는 기능을 추가할 수 있다.
	\item [MCU 변경] ARDUINO NANO로 펌웨어를 개발할 때 메모리 용량이 작아 어려움을 겪었다. STM32L432KC 같은 는 ARDUINO NANO보다 성능이 좋은 Arduino로, 방향만 반대일 뿐 핀의 순서와 종류가 모두 같아 ARDUINO NANO에 넣었던 펌웨어를 그대로 사용할 수 있다.
	\item [Heating system 추가] 천체 관측시 렌즈에 이슬이 맺혀 관측에 어려움을 겪는 경우가 종종 있다. 이를 해결하기 위하여 날씨가 추운 날에는 모터가 얼어서 돌아가지 않거나 렌즈에 서리가 껴서 초점이 맞아도 맞지 않은 것으로 판단할 수 있다. 따라서 이를 예방하기 위하여 열선을 깔아서 DHT22에서 측정한 온도를 바탕으로 특정한 온도 이하로 내려가게 된다면 열선이 활성화될 수 있게 할 수 있다.
	\item [EEPROM 활용] EEPROM은 Arduino 내부에 저장된 비휘발성 메모리로, 컴퓨터의 ‘RAM’과 같은 역할을 하고 있다. 비휘발성이기 때문에 Arduino를 초기화하거나 껐다가 다시 켰더라도 정보를 저장하고 있다. 
	Arduino별로 한 EEPROM의 주소에 들어갈 수 있는 수의 크기가 달라진다. ARDUINO NANO는 4KB의 EEPROM을 지원하므로 0~255까지의 수를 한 번에 저장할 수 있다. 이렇게 저장할 수 있는 수가 작으므로, 여러 가지 주소를 활용하여 큰 수 또한 나타낼 수 있다.(수를 진법으로 바꾸는 과정과 유사함) 실제로 이를 기반으로 펌웨어를 제작하여 보았지만, 수가 약 32000 이상으로 넘어가는 상황에서는 갑자기 수가 이상하게 커지는 오류가 발견되었고, 이를 고쳐야 할 것이다.
	\item [모터 연결 상태 체크 기능 추가] 모터의 연결 상태는 펌웨어를 실행하는 데 아주 중요하다. 만약 펌웨어가 실행되는 도중에 모터가 연결되지 않으면, 스위치를 움직였을 때 스위치의 숫자는 움직이지만, 모터는 움직이지 않아 결과적으로 숫자의 오류를 불러일으킨다. 또한, 모터를 펌웨어가 실행되는 도중에 연결선을 뽑으면 펌웨어에 에러가 일어나는데, 이 경우 다시 모터를 꽂더라도 정상적으로 실행이 되지 않는다. 따라서 이런 여러 상황에 대하여 모터의 연결 상태를 대비한 에러 코드를 설정해야 숫자와 모터가 오차를 일으키는 일이 없을 것이다.
\end{description}

또한 GS-touch를 활용하여 태양, 행성, 달 등 점 광원이 아닌 천체의 초점 조절 알고리즘을 개발할 생각이다. 앞서 언급한 것처럼 별은 점 광원이므로 FWHM이나 HFD를 이용하여 초점 조절하는 알고리즘이 개발되어 있으나 태양, 행성, 달 등 점 광원이 아닌 천체는 FWHM이나 HDF를 이용한 알고리즘을 적용하기 힘들다. GS-touch를 이용하면 모터 포커서를 정밀하게 제어할 있으므로 태양, 행성, 달 등을 관측할 때 각각의 천체에 적합한 자동 초점 조절 알고리즘 구현하는데 활용할 수 있을 것으로 생각된다. 
