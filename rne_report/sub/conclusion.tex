\section{결론 및 제언}
	
\subsection{결론}

 본 연구를 통하여 모터 초점 조절 장치 및 펌웨어 개발, 무선통신 개발, ASCOM 드라이버 개발 등을 통하여 자동 초점 조절 장치 개발을 위한 기반을 마련하였다. 이를 통하여 기존에 있던 초점 조절 장치의 성능을 무선통신 등의 여러 면으로 개선하여 사용자들에게 편의성을 제공하였다. 본 연구가 나아간다면 성능이 좋고 여러 가지 기능이 존재하는 자동 초점 조절 장치를 개발할 수 있을 것이다. 이와 관련된 자세한 내용은 이후 문서에서 다룬다.


\subsection{제언}



\subsection{추후 연구}


\subsubsection{태양, 행성, 달 등의 자동초점조절 알고리즘 개발}

앞서 언급한 것처럼 별은 점 광원이므로 FWHM이나 HFD 등을 이용하여 초점을 정확하게 조절할 수 있으나 태양, 행성, 달 등 점 광원이 아닌 천체는 그 경계선을 이용하여 초점을 판단할 수 있을 것이다. 


\subsubsection{기능 개선}

현재 개발된 GSfocus는 초보 단계로서 개선할 수 있는 부분들이 아직 많이 있다. 

\begin{description}[font=$\bullet$~\normalfont\scshape\color{red!50!black}]
	\item [MCU 변경] STM32L432KC는 ARDUINO NANO보다 성능이 좋은 Arduino로, 방향만 반대일 뿐 핀의 순서와 종류가 모두 같아 ARDUINO NANO에 넣었던 펌웨어를 그대로 사용할 수 있다.
	\item [Heating system 추가] 날씨가 추운 날에는 모터가 얼어서 돌아가지 않거나 렌즈에 서리가 껴서 초점이 맞아도 맞지 않은 것으로 판단할 수 있다. 따라서 이를 예방하기 위하여 열선을 깔아서 DHT22에서 측정한 온도를 바탕으로 특정한 온도 이하로 내려가게 된다면 열선이 활성화될 수 있게 할 수 있다.
	\item [EEPROM 활용] EEPROM은 Arduino 내부에 저장된 비휘발성 메모리로, 컴퓨터의 ‘RAM’과 같은 역할을 하고 있다. 비휘발성이기 때문에 Arduino를 초기화하거나 껐다가 다시 켰더라도 정보를 저장하고 있다. 
	Arduino별로 한 EEPROM의 주소에 들어갈 수 있는 수의 크기가 달라진다. ARDUINO NANO는 4KB의 EEPROM을 지원하므로 0~255까지의 수를 한 번에 저장할 수 있다. 이렇게 저장할 수 있는 수가 작으므로, 여러 가지 주소를 활용하여 큰 수 또한 나타낼 수 있다.(수를 진법으로 바꾸는 과정과 유사함) 실제로 이를 기반으로 펌웨어를 제작하여 보았지만, 수가 약 32000 이상으로 넘어가는 상황에서는 갑자기 수가 이상하게 커지는 오류가 발견되었고, 이를 고쳐야 할 것이다.
	\item [모터 연결 상태 체크 기능 추가] 모터의 연결 상태는 펌웨어를 실행하는 데 아주 중요하다. 만약 펌웨어가 실행되는 도중에 모터가 연결되지 않으면, 스위치를 움직였을 때 스위치의 숫자는 움직이지만, 모터는 움직이지 않아 결과적으로 숫자의 오류를 불러일으킨다. 또한, 모터를 펌웨어가 실행되는 도중에 연결선을 뽑으면 펌웨어에 에러가 일어나는데, 이 경우 다시 모터를 꽂더라도 정상적으로 실행이 되지 않는다. 따라서 이런 여러 상황에 대하여 모터의 연결 상태를 대비한 에러 코드를 설정해야 숫자와 모터가 오차를 일으키는 일이 없을 것이다.
\end{description}
