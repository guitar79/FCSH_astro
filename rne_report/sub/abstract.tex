%\maketitle  % command to print the title page with above variables
\makecover  % command to print the title page with above variables

\setcounter{page}{1}
\renewcommand{\thepage}{\roman{page}}

%----------------------------------------------
%   Table of Contents (자동 작성됨)
%----------------------------------------------
\cleardoublepage
\addcontentsline{toc}{section}{Contents}
\setcounter{secnumdepth}{3} % organisational level that receives a numbers
\setcounter{tocdepth}{3}    % print table of contents for level 3
\baselineskip=2.2em
\tableofcontents


%----------------------------------------------
%     List of Figures/Tables (자동 작성됨)
%----------------------------------------------
\cleardoublepage
\clearpage
\listoffigures	% 그림 목록과 캡션을 출력한다. 만약 논문에 그림이 없다면 이 줄의 맨 앞에 %기호를 넣어서 코멘트 처리한다.

\cleardoublepage
\clearpage
\listoftables  % 표 목록과 캡션을 출력한다. 만약 논문에 표가 없다면 이 줄의 맨 앞에 %기호를 넣어서 코멘트 처리한다.


\cleardoublepage
\clearpage

%---------------------------------------------------------------------
%                  영문 초록을 입력하시오
%---------------------------------------------------------------------
%\begin{abstracts}     %this creates the heading for the abstract page
%	\addcontentsline{toc}{section}{Abstract}  %%% TOC에 표시
%	\noindent{
%			Put your abstract here. Once upon a time, \gshs said : `The first, and the best.'
%	}
%\end{abstracts}

%\cleardoublepage
%\clearpage

\begin{abstractkor}
	%\addcontentsline{toc}{section}{초록}  %%% TOC에 표시
	\noindent{본 연구에서 천체 망원경 모터 포커서 컨트롤러인 GS-touch를 개발하였다. GS-touch\는 $12 \textrm{V}$로 구동되며 2상 바이폴라 스테핑 모터가 장착된 모터 포커서를 구동할 수 있다. GS-touch의 전원은 DC 12V를 사용하며, 4개의 버튼으로 메뉴를 선택하거나 모터를 정방향 역방향으로 회전시킬 수 있으며 OLED (Organic Light-Emitting Diode) 디스플레이어에 표시되는 메뉴를 조정할 수 있다. DHT22를 이용하여 온도, 습도 값을 읽어들여 OLED 디스플레이어에 표현할 수 있으며 온도와 습도 값은 관측 환경을 저장기록하는데 이용할 수 있다. GS-touch 전용의 ASCOM 드라이버를 개발하여 GS-touch를 PC에 연결하여 ASCOM을 지원하는 MaximDL, FocusMAX 등의 소프트웨어로 구동하여 자동 초점 조절 알고리즘을 구현하는데 이용될 수 있다. 
		}
\end{abstractkor}







