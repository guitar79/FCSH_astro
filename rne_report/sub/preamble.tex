\documentclass{new_report}
\graphicspath{{images/}}
% 이곳에 필요한 별도의 패키지들을 적어넣으시오.
\usepackage{verbatim} % for commment, verbatim environment
\usepackage{spverbatim} % automatic linebreak verbatim environment
%\usepacakge{indentfirst}
\usepackage{tikz}
\usepackage{pgfplots}
\usetikzlibrary{patterns}
%\tikzset{
%	image label/.style={
%		every node/.style={
%fill=black,
%text=white,
%			font=\sffamily\scriptsize,
%			anchor=south west,
%			xshift=0,
%			yshift=0,
%			at={(0,0)}
%		}
%	}
%}
\usepackage{amsmath}
\usepackage{amsfonts}
\usepackage{amssymb}
\usepackage{float}
\usepackage{graphicx}
\usepackage{tabularx}
\usepackage{multirow}
\usepackage{booktabs}
\usepackage{longtable}
\usepackage{gensymb}
%\usepackage{subfigure}
\usepackage{subcaption}
%\usepackage{floatrow}
%\usepackage{pict2e}

\usepackage{pgfplots}
\pgfplotsset{
	compat=newest,
	label style={font=\sffamily\scriptsize},
	ticklabel style={font=\sffamily\scriptsize},
	legend style={font=\sffamily\tiny},
	major tick length=0.1cm,
	minor tick length=0.05cm,
	every x tick/.style={black},
}

\usetikzlibrary{shapes}
\usetikzlibrary{plotmarks}
\usepackage{listings}
\usepackage{hologo}
\usepackage{makecell}

\lstset{
	basicstyle=\small\ttfamily,
	columns=flexible,
	breaklines=true
}

\citation
\bibdata

%: ----------------------------------------------------------------------
%:               보고서 정보를 입력하시오
% ----------------------------------------------------------------------
% 아래와 같은 command를 만들면 길이가 긴 용어를 간편하게 사용할 수 있습니다. 단, 이미 지정된 함수명들은 새로운 함수명으로 사용할 수 없습니다.
\newcommand{\gshs}{Gyeonggi Science High School for the Gifted }

\researchtype{기초} % 기초 / 심화
\reporttype{결과} % 중간 / 결과

\title{천체망원경 모터 포커서 컨트롤러 구동 시스템 및 ASCOM 드라이버 개발 } % 제목 개행 시 \linebreak 사용. \\나 \newline 은 안됨.
\englishtitle{English title}% 제목 개행 시 \linebreak 사용. \\나 \newline 은 안됨.

\author[1] {} % 제 1 저자명
\email[1]{} % 제 1 저자 이메일
\author[2] {} % 제 2 저자명
\email[2]{} % 제 2 저자 이메일
\advisor{} % 지도교사명
\advisorEmail{} % 지도교사 이메일

%%%%%%%%%%%%%%%%%%%%%%%%%%%%%%%%%%%%%%%%%%%%%%%
%%%% researchtype이 '심화'일 경우에만 나타남 %%%%
\professor{} % 지도교수명
\professorEmail{} % 지도교수 이메일
%%%%%%%%%%%%%%%%%%%%%%%%%%%%%%%%%%%%%%%%%%%%%%%%
\summitdate{}{}{} % 제출일 (연, 월, 일)
\newtheorem{definition}{정의}
