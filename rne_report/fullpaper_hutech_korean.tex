% Humantech paper award tex file
% Initiated by 목진욱(경기과학고등학교 물리과 전문교원)
\documentclass{fullpaper_hutech}
\renewcommand\refname{참고문헌}
\begin{document}
\title{제목}
\twocolumn[
\begin{@twocolumnfalse}
\begin{flushleft}
\maketitle
\begin{abstract}
\textbf{(Abstract) 휴먼테크논문대상 지원자는 공저자를 포함하여 국내외 고교, 대학(원) 재학중인 한국인 및 국내 대학(원)에 재학중인 외국인 학생에 한정한다. 회사, 연구소 等 기관에 재직중인 학술연수자 또는 Part-time 학생은 응모대상에서 제외된다. 논문접수 마감일 이전에 온라인을 포함하여 국내외 공개 출판물에 발표되지 않은 내용이어야 한다. 휴먼테크논문대상은 미래 과학한국을 이끌어갈 창의적이고 도전적인 젊은이들을 발굴하고 학교 내 연구 분위기 활성화와 기술을 중시하는 사회 분위기 조성에 목적이 있다. 심사의 공정성을 위해 초록과 논문에는 절대로 저자의 이름, 전공, 학교명, 학교 로고 혹은 지도교사/교수 이름을 기재할 수 없다. 시상금은 주저자 1인에게 지급되며, 시상금 및 세금(삼성전자 부담)은 주저자의 소득금액으로 부과된다. 초록과 논문은 반드시 해당 접수기간에 휴먼테크논문대상 홈페이지를 통해 제출해야 한다. 그리고 초록과 논문 응모 시에는 규정된 양식을 사용하고 분량을 준수해야 한다. 논문은 12장 이내로 작성을 권장한다. 국문 서체는 바탕체, 영문 서체는 Times New Roman, 글자크기 10 point로 작성한다.}
\end{abstract}
\vspace{20pt}
\end{flushleft}
\end{@twocolumnfalse}
]
\thispagestyle{firstpage}


\section{서론}
휴먼테크논문대상은 미래 과학한국을 이끌어갈 창의적이고 도전적인 젊은이들을 발굴하고 학교 내 연구 분위기 활성화와 기술을 중시하는 사회 분위기 조성을 위해 1994년에 제정되었다. 휴먼테크논문대상 지원자는 국내외 고교, 대학(원) 재학중인 한국인 및 국내 대학(원)에 재학중인 외국인 학생에 한한다. 논문은 논문접수 마감일 이전에 온라인을 포함하여 발표되지 않은 내용이어야 한다. 휴먼테크논문대상은 3단계로 평가가 진행된다. 1차는 초록심사, 2차는 논문심사, 3차는 발표심사가 진행된다. 각 분야의 전문가가 제출된 초록과 논문을 심사한다. 초록과 논문 작성은 목적, 범위, 결과, 중요성, 독창성에 대해 명확히 해야 한다. 

\section{작성 양식}
초록 및 논문의 양식은 A4용지(21cm×29.7cm)를 사용하며, 용지 여백은 위 3cm, 아래 2.5cm, 왼쪽 1.5cm, 오른쪽 1.5cm, 머리글 2cm, 바닥글 1cm로 한다. 머리글과 바닥글의 경우, ``짝수 페이지와 홀수 페이지가 다르게'', ``첫 페이지가 다르게''로 지정한다.

머리글에는 ``$23^{\rm rd}$ HumanTech Paper Award''를 표기하되, 첫 페이지에는 12pt(굵게), 이후 페이지에는 9pt로 하되 짝수 페이지는 왼쪽 정렬, 홀수 페이지는 오른쪽 정렬로 한다. (첫 페이지는 왼쪽 정렬) 바닥글에는 페이지 번호를 표시하되, 짝수 페이지는 왼쪽, 홀수 페이지는 오른쪽에 표시한다.

논문은 12장이내로 작성을 권장한다. 초록 및 논문의 제목은 2줄을 넘지 않게 하고, Abstract는 15줄을 넘지 않게 한다.

초록 및 논문은 영문 또는 국문으로 작성하여야 한다. 영문 서체는 Times New Roman, 국문 서체는 바탕체로, 줄간격 1.0으로 하며, 제목과 Abstract는 왼쪽 정렬, 본문 내용은 양쪽 정렬로 한다. 글씨 크기는 제목 20pt(굵게), Abstract 10pt(굵게), 본문의 제목은 11pt(굵게), 내용은 10pt, 그림 및 도표 9pt(굵게), 참고문헌은 9pt로 한다. 

심사의 공정성을 위해 초록과 논문에는 절대로 저자의 이름, 전공, 학교명, 학교 로고 혹은 지도교사/교수 이름을 기재할 수 없다.

본문은 다단 편집(단수 2단, 단 간격 0.5cm)으로 하며, 본문 내의 제목, 표, 그림, 수식은 위, 아래로 각 각 한 줄을 띄어 구분한다.

본문 내용의 목차 표시는 다음과 같이 작성한다. 장: 1. , 2. , 3. $ \sim $ , 절: 1.1. , 2.1. $ \sim $.

모든 단위는 SI 단위 사용을 원칙으로 한다. 약어가 처음 나타나는 경우는 문자를 생략하지 않고 전부 써주어야 하고 만약 비표준 약어를 사용하는 경우에는 명확하게 정의되어야 한다.

\section{도움말}

\subsection{그림과 표}

큰 그림과 표는 두개의 단에 이어진다. 그림 설명은 그림 아래에, 표 제목은 표 위에 작성한다. 2개이상의 그림이 함께 있다면, ``(a)'' 와 ``(b)''로 표시하여 구분한다. 논문 내용에서 언급된 그림과 표가 실제로 존재하는지 확인한다. 

그림은 ``Fig.''로 표는 ``Table''로 표기하고 아라비아 숫자로 번호를 매긴다. 
그림의 축 레이블은 혼동되기 쉬워 심볼 대신 단어로 표기한다. 예를 들어, 단순히 ``M.''으로 작성하지 않고, ``Magnetization,'' 또는 ``Magnetization, M,''으로 작성하고 괄호 안에 단위를 적는다. 축 레이블에 단위만 작성하지 않는다. 예로 Fig. \ref{Fig01}\를 보면, ``A/m.''으로만 작성하지 않고, ``Magnetization (A/m)'' 또는 ``Magnetization (A m ),''으로 작성한다. 축 레이블을 ``Temperature (K),''와 같이 작성하고, ``Temperature/K.''와 같은 단위와 수량의 비율로 작성하지 않는다. 

Multipliers can be especially confusing. Write ``Magnetization ($\rm kA/m$)'' or ``Magnetization ($\rm A/m$).''
\footnote{양식 디자이너 주 : 한글 양식에는 없는 텍스트이다.}

\begin{table}[t]
\fontsize{9}{9}\selectfont
\begin{center}
\caption{{\bf Units for Magnetic Properties}\\(Gaussian units are the same as cgs emu for magnet ostatics; $\rm Mx=\textrm{maxwell}$, $\rm G=\textrm{gauss}$, $\rm Oe=\textrm{oersted}$; $\rm Wb=\textrm{weber}$, $\rm V=\textrm{volt}$, $\rm s=\textrm{second}$, $\rm T=\textrm{tesla}$, $\rm m=\textrm{meter}$, $\rm A=\textrm{ampere}$, $\rm J=\textrm{joule}$, $\rm kg=\textrm{kilogram}$, $\rm H=\textrm{henry}$.)}\label{Table01}
\begin{tabular*}{\columnwidth}{lll}
\specialrule{1.5pt}{0pt}{4pt}
\multicolumn{1}{c}{Symbol}&\multicolumn{1}{c}{Quantity}&\multicolumn{1}{c}{\begin{tabular}[c]{c@{}c@{}}Conversion from Gaussian\\and cgs emu to SI\end{tabular}}\\
\specialrule{0.5pt}{4pt}{4pt}
$\Phi$&magnetic flux&$1\,\rm Mx\ \rightarrow\ 10^{-8}\,Wb=10^{-8}\,V\mycdot s$\\
$B$&\multicolumn{1}{l}{\begin{tabular}[t]{@{}l@{}}magnetic flux density,\\magnetic induction\end{tabular}}&$1\,\rm G\,\rightarrow\,10^{-4}\,T=10^{-4}\,Wb/m^2$\\
$H$&magnetic field strength&$1\,\rm Oe\,\rightarrow\,10^3 /(4\pi)\,A/m$\\
$m$&magnetic moment&\multicolumn{1}{l}{\begin{tabular}[t]{@{}l@{}}$1\,\rm egr/G=1\,emu$ \\$\rightarrow$ $10^{-3}\,\rm A\mycdot m^2 =10^{-3}\,J/T$\end{tabular}} \\
$M$&magnetization&\multicolumn{1}{l}{\begin{tabular}[t]{@{}l@{}}$1\,\rm erg/(G\mycdot cm^3 )=1\,emu/cm^3$\\$\rightarrow$ $10^3\,\rm A/m$\end{tabular}} \\
$4\pi M$&magnetization&$1\,\rm G$ $\rightarrow$ $10^3 /(4\pi)\,\rm A/m$ \\
$\sigma$&\multicolumn{1}{l}{\begin{tabular}[t]{@{}l@{}}mass magnetization,\\specific magnetization\end{tabular}} &\multicolumn{1}{l}{\begin{tabular}[t]{@{}l@{}}$1\,\rm erg/(G\mycdot g)=1\,emu/g$\\$\rightarrow$ $1\,\rm A\mycdot m^2 /kg$\end{tabular}} \\
$j$&\multicolumn{1}{l}{\begin{tabular}[t]{@{}l@{}}magnetic dipole\\moment\end{tabular}}&\multicolumn{1}{l}{\begin{tabular}[t]{@{}l@{}}$1\,\rm erg/G=1\,emu$\\$\rightarrow$ $4\pi\times 10^{-10}\,\rm Wb\mycdot m$\end{tabular}} \\
$J$&magnetic polarization&\multicolumn{1}{l}{\begin{tabular}[t]{@{}l@{}}$1\,\rm erg/(G\mycdot cm^3 )=1\,emu/cm^3$\\$\rightarrow$ $4\pi\times 10^{-4}\,\rm T$\end{tabular}} \\
$\chi$, $\kappa$&susceptibility&$1$ $\rightarrow$ $4\pi$\\
$\chi_{\rho}$&mass susceptibility&$1\,\rm cm^3 /g$ $\rightarrow$ $4\pi\times 10^{-3}\,\rm m^3 /kg$\\
$\mu$&permeability&\multicolumn{1}{l}{\begin{tabular}[t]{@{}l@{}}$1$ $\rightarrow$ $4\pi\times 10^{-7}\,\rm H/m$\\$=4\pi\times 10^{-7}\,\rm Wb/(A\mycdot m)$\end{tabular}} \\
$\mu_r$&relative permeability&$\mu$ $\rightarrow$ $\mu_r$\\
$w$, $W$&energy density&$1\,\rm erg/cm^3$ $\rightarrow$ $10^{-1}\,\rm J/m^3$\\
$N$, $D$&demagnetizing factor&$1$ $\rightarrow$ $1/(4\pi)$\\
\specialrule{1.5pt}{4pt}{-7pt}
\end{tabular*}
\end{center}
\quad No vertical lines in table. Statements that serve as captions for the entire table do not need footnote letters.
\end{table}

\begin{figure}[t]
\begin{center}
\includegraphics[width=\columnwidth]{example-image-a}
\end{center}
\caption{{\bf Magnetization as a function of applied field.} Note that ``Fig.'' is abbreviated. There is a period after the figure number. It is good practice to explain the significance of the figure in the caption.}
\label{Fig01}
\end{figure}



\subsection{참고문헌}

번호는 대괄호 안에 연속하여 작성한다 \cite{True00}. 마침표는 대괄호 뒤에 붙는다 \cite{Schluter00}. 인용이 여러개인 경우 \cite{Schluter00}, \cite{Plazzo11} 또는 \cite{True00,Schluter00,Plazzo11}\으로 작성한다. 책의 부분을 인용하면 관련 페이지 번호를 적는다 \cite{Schluter00}. 문장에서 참조를 언급할 때 \cite{Plazzo11}으로 표기하고, ``Ref. \cite{Plazzo11}'' 또는 ``reference \cite{Plazzo11}''\로 작성하지 않는다. 단, 문장 처음인 경우는 ``참고 \cite{Plazzo11}\은 ... .''로 표현한다.

참고문헌은 다음과 같은 방법으로 작성한다. 저자명은 성을 먼저 쓰고 콤마로 구분한 뒤 이름의 이니셜을 표기한다. Article의 Title은 Full Title을 정확하게 표기하며, 첫자는 대문자로 쓴다. Journal 및 단행본의 제목은 이탤릭체로 표기하며, 모든 단어의 첫자는 대문자로 쓴다. Journal Title의 경우, 일반적인 용례에 따라 약어로 표기할 수 있다.  Vol.은 굵게(Bold)하며, 단행본의 경우 출판사명과 출판지(지역명)를 표기한다. web-only journal의 경우, 위에서 제시한 기본 정보와 함께 전체 URL 혹은 DOI를 표기한다. website의 경우, author, 인용 페이지의 title, URL 및 posting 연도를 표기한다.  발행연도(posting 연도)는 괄호 안에 표기한다.

\subsection{약어와 頭문자어}

논문에서 약어와 頭문자어가 처음 나올 때는 그 정의를 표기한다.

\subsection{수식}

수식은 \eqref{eq01}\와 같이 괄호 안에 번호를 순차적으로 적어 수식 오른쪽에 작성한다. 수식은 수식편집기를 사용하여 작성한다. 

To make your equations more compact, you may use the solidus ( / ), the exp function, or appropriate exponents. Use parentheses to avoid ambiguities in denominators. Punctuate equations when they are part of a sentence, as in
\begin{equation}\label{eq01}
\begin{split}\int_0^{r_2} &F(r,\varphi)\,dr\,d\varphi=[\sigma r_2 /(2\mu_0 )] \\
&\times\int_0^{\infty} \exp(-\lambda |z_j -z_i |)\lambda^{-1} J_1 (\lambda r_2 )J_0 (\lambda r_i )d\lambda.\end{split}
\end{equation}

문장에서 수식을 언급할 때 \eqref{eq01}\으로 표기하고, ``Eq. \eqref{eq01}'' 또는 ``equation \eqref{eq01},''\로 작성하지 않는다. 단, 문장 처음인 경우는 ``수식 \eqref{eq01}\은 ... ."으로 표현한다.


\subsection{그 외 권고사항}

마침표와 콜론(:) 뒤에 한 칸을 띄어쓰기 한다. 소수점 앞에는 ``0''을 사용: ``0.25.'' 샘플의 크기는 ``0.1cm×0.2cm''로 표기한다. 시간 단위 ``초''의 약어는 ``sec''가 아닌 ``s''이다.  단위를 표현할 때 약어와 단어를 모두 표기하는 것을 혼용하지 않는다. 예를 들어, ``Wb/m'' 또는 ``webers per square meter''으로 표기하고 ``webers/m''와 같이 표기하지 않는다. 수의 범위를 표현할 때는 ``7$ \sim $9''가 아닌 ``7 에서 9'' 또는 ``7-9''으로 표기한다.

문장이 닫힌 괄호로 끝나면 마침표는 괄호 밖에 위치한다 (이것처럼). (괄호 안에서 문장이 끝나면 괄호 안에 마침표가 위치한다.)

\section{결론}

초록 및 논문의 전체 구성은 다음과 같다.
①제목 ②Abstract ③본문 ④참고 문헌.

\begin{thebibliography}{99}
\bibitem{True00} True, H. L. and Lindquist, S. L. A. A yeast prion provides a mechanism for genetic variation and phenotypic diversity. {\it Nature} {\bf 407}, 477--483 (2000)
\bibitem{Schluter00} Schluter, D. {\it The Ecology of Adaptive Radiation} (Oxford Univ. Press, 2000)
\bibitem{Plazzo11} Plazzo, A. P. et al. Bioinformation and mutational analysis of channelrhodopsin-2 cation conducting pathway. {\it J. Biol. Chem.} http://dx.doi.org/10.1074/jbc.M111.326207 (2011)
\end{thebibliography}


\end{document}