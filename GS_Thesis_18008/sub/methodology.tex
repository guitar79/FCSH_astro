\section{연구 과정 및 결과}

\subsection{샘플 제작}
본 연구에서는 간단하고 빠르게 페로브스카이트 결정을 만들 수 있는 PDMS stamping 방법을 사용하였다. 다음은 PDMS stamping 방법으로 페로브스카이트 결정 샘플을 제작하는 과정이다.
\begin{enumerate}
	\item $\rm CsPbBr_3$을 만들기 위해 $\rm CsBr$과 $\rm PbBr_2$를 1:1의 몰 비율로 섞고 용매는 DMSO(Dimethyl Sulfoxide)를 사용하였다.
	\item Sonication을 이용해서 용매와 용질을 균일하게 섞어주었다.
	\item Silicon wafer 위에 제조된 용액을 스포이트를 이용해서 떨어뜨린 뒤, 2,000rpm으로 1분간 회전시키는 spin coating을 이용하여 균일하게 펼쳐주었다.
	\item 섭씨 100도로 달궈놓은 핫플레이트에서 silicon wafer를 5분간 PDMS로 눌러주었다.
	\begin{figure}[H]
		\begin{center}
			\begin{tabular}{ccc}
				\includegraphics[width=0.3\textwidth]{sonicator}&
				\includegraphics[width=0.3\textwidth]{spin_coating}&
				\includegraphics[width=0.3\textwidth]{hotplate}
			\end{tabular}
		\end{center}
		\begin{tikzpicture} [remember picture,overlay]
		\node[text=white] at (0.6, 4.7) {(a)};
		\node at (5.2, 4.7) {(b)};
		\node[text=white] at (10.1, 4.7) {(c)};
		\end{tikzpicture}	
		\caption{Sample production. (a) Sonicating, (b) Spin coating, (c) PDMS stamping on hot plate.}
		\label{fig:sample}  
	\end{figure}
	\item 결정이 생겼는지 광학현미경을 통해서 확인한 뒤(Figure 6), 가장 잘 형성된 결정에 405(nm) 파장의 레이저를 사용해서 PL 촬영을 진행하였다. 
	\begin{figure}[H]
		\begin{center}
			\begin{tabular}{cc}
				\includegraphics[width=0.45\textwidth]{OM}
			\end{tabular}
		\end{center}
		\caption{A silicon wafer taken with an OM (optical microscope).}
		\label{fig:om}  
	\end{figure}
\end{enumerate}
Perovskite는 70도 이상의 온도에서 빠른 degradation이 나타나는 것으로 알려져 있다. 본 실험에서는 100도에서 PDMS stamping을 진행하였는데 온도가 높아도 결정이 비교적 적게 생기긴 하지만 PL peak의 위치는 변하지 않기 때문에 그렇게 진행하였다.


\subsection{데이터 추출}
제작된 sample을 NT-MDT Spectrum Instruments 사의 Ntegra 기기를 통하여 $75(\micro m)\times75(\micro m)$의 영역을 PL mapping 하였다. 생성된 단결정에 측정할 위치를 정해 놓고 PL을 측정하였다.  PL mapping이란 특정 영역에서의 모든 PL 데이터를 얻는 기법으로 전체적인 특성을 한눈에 볼 수 있다는 장점이 있다. 이 데이터는 레이저의 조리개를 $OD = 2$ 로 맞춰놓은 ND2 상태에서 측정하였다. 이렇게 만들어진 파일에서는 임의의 점에서의 PL data를 얻어낼 수 있다는 장점이 있다.
\begin{figure}[H]
	\begin{center}
		\begin{tabular}{cc}
			\includegraphics[width=0.65\textwidth]{Nova_screen_capture}&
			\includegraphics[width=0.4\textwidth]{line123}
		\end{tabular}
	\end{center}
	\begin{tikzpicture} [remember picture,overlay]
	\node at (0.6, 6.2) {(a)};
	\node[text=white] at (10.5, 6) {(b)};
	\end{tikzpicture}
	\caption{Extracting data and line setting. (a) Nova-Px program for extracting data, (b) line setting.}
	\label{fig:nova}  
\end{figure}
Nova-Px 프로그램을 활용하여 PL mapping 된 파일에서 데이터를 각 점별로 뽑아내었다. Figure \ref{fig:nova}\와 같은 화면에서 십자의 위치를 조절하여 원하는 위치의 PL peak을 얻어낼 수 있다. 중앙에서부터 바깥으로 나갈 때의 PL peak의 경향성을 알아보기 위해 Figure \ref{fig:nova}의 오른쪽 사진에서 볼 수 있는 1, 2, 3 경로로 이동하며 PL peak 자료를 추출하였다.

Figure 7의 (b)에서 그림 상으로는 정중앙이 아닐 수 있지만, PL peak이 가장 높게 나온 곳이므로 올바른 경향성을 찾아내기 위하여 중앙을 대표하는 기준점으로 선정하였다. 기준점의 사진상 좌표는 (59.0, 53.6)이고, 선정된 기준점으로부터 바깥 방향으로 나가는 경로 1, 2, 3위의 관측점을 Table 1과 같이 설정하였다.

\begin{table}[H]%[width=1.0\linewidth]
	\caption{Routing lines 1, 2, and 3}
	\label{table01}
	\centering
	\begin{tabular}{c c}
	\toprule
	경로 번호 & 경로\\
	\toprule
	Line 1 & (59.0, 53.6, 33)-->(62.3, 56.9, 14) / (+0.4, +0.4) 씩 이동, 9개소 관측\\
	Line 2 & (59.0, 53.6, 33)-->(68.0, 51.3, 13) / (+0.8, -0.2) 씩 이동, 12개소 관측\\
	Line 3 & (59.0, 53.6, 33)-->(64.7, 47.9, 17) / (+0.4, -0.4) 씩 이동, 15개소 관측\\
	\toprule
	\end{tabular}
\end{table}
중앙으로 잡은 점을 point 0, 각 line에 대해 있는 점들을 point 1-1, 1-2, … , 1-8, 2-1, 2-2, … , 2-11, 3-1, 3-2, … , 3-14로 정의하였다. line 1은 point 0부터 point 1-8, line 2 은 point 0 부터 point 2-11, line 3은 point 0부터 point 3-14까지 이다. 
\\

\subsection{분석 과정}
\subsubsection{Point data peak fitting}
각 점의 추출된 data를 분석하기 위해서 Origin 9 프로그램을 사용하였다. Chen (2018) 에 의하면 $\rm CsPbBr_3$에서 biexciton과 exciton의 peak\이 나타나는 wave length\는 각각 약 580nm, 600nm 이다\cite{chen2018room}. 이 사실을 바탕으로 PL data에서 보인 peak\을 두 개의 peak의 합으로 fitting 하였다. Peak fitting\을 할 때 Hartley (1961)의 Gauss fitting 메커니즘을 프로그램에서 사용하였으며, biexciton 과 exciton이 존재하는 wavelength에 peak 위치를 설정한 후 fitting\을 진행하였다\cite{hartley1961modified}. Figure \ref{fig:point0}은 그중 하나의 예시이다.

\begin{figure}[H]
	\begin{center}
		\begin{tabular}{cc}
			\includegraphics[width=0.8\textwidth]{point0}
		\end{tabular}
	\end{center}
	\caption{The PL data of the set point 0 is shown by sum of exciton and biexciton peak.}
	\label{fig:point0}  
\end{figure}

Figure \ref{fig:point0}\과 같이 multiple peak fitting을 마친 후에는 각 peak의 x값, 즉 wavelength 값과 y값, 즉 intensity 값을 데이터로 기록한 후 분석하였다.

\subsubsection{Line data analysis}
위의 과정에서 각 point 들의 data에 대한 peak fitting을 한 이후에 그 경향성을 보기 위해 필요한 과정이다. 분석하고자 하는 것은 중앙에서 바깥으로 가면서 peak intensity의 경향성이다. 이를 위해서 peak fitting 과정에서 얻은 데이터인 각 point에서의 biexciton, exciton peak의 intensity값을 y축, point 번호를 x 축으로 설정하여  line 1, line 2, line 3 별로 막대그래프를 그려서 경향성을 볼 수 있었다.
\\

\subsection{분석 결과 및 해석}
Line 1, Line 2, Line 3 에서의 결과를 각각 Figure \ref{fig:line1}, Figure \ref{fig:line2}, Figure \ref{fig:line3}에 나타내었다.

\begin{figure}[H]
	\begin{tabular}{cc}
		\includegraphics[width=0.3\textwidth]{line1}
		\begin{tikzpicture} [remember picture,overlay]	
		\node[text=white] at (-4, 4) {(a)};
		\end{tikzpicture}
		&
		\begin{tikzpicture}
		\begin{axis} [
		width=0.70\textwidth,%
		height = 4cm,%
		ybar,%
		bar width=5pt,
		title={Line 1},%
		xtick = data,%
		symbolic x coords={0, 1, 2, 3, 4, 5, 6, 7, 8},%
		xlabel= {Viewpoint},%
		ylabel= {Intensity(a.u.)},%
		ymin=0,ystep=5000,ymax=35000.0,%
		scaled y ticks = false,%
		ymajorgrids = true,
		legend style={at={(0.02,10)}},legend pos=north east]%
		\addplot table [x=no, y=biexciton] {./data/line1.csv}; %
		\addlegendentry {biexciton}%
		\addplot table [x=no, y=exciton] {./data/line1.csv}; %
		\addlegendentry {exciton}%
		\end{axis}
		\node at (-0.9, 2.9) {(b)};
		\end{tikzpicture}
	\end{tabular}
	\caption{(a) Route set to line 1, (b) Analyzed data :tendency in the path along line 1.}
	\label{fig:line1}  
\end{figure}




Figure \ref{fig:line1}, 즉 line 1에서는 exciton과 biexciton 모두 감소하는 추세를 보이다가 끝에서 증가하는 모습을 볼 수 있다.

\begin{figure}[H]
	\begin{tabular}{cc}
		\includegraphics[width=0.3\textwidth]{line2}
		\begin{tikzpicture} [remember picture,overlay]	
		\node[text=white] at (-4, 4) {(a)};
		\end{tikzpicture}
		&
		\begin{tikzpicture}
		\begin{axis} [
		width=0.70\textwidth,%
		height = 4cm,%
		ybar,%
		bar width=5pt,
		title={Line 2},%
		xtick = data,%
		symbolic x coords={0, 1, 2, 3, 4, 5, 6, 7, 8, 9, 10, 11},%
		xlabel= {Viewpoint},%
		ylabel= {Intensity(a.u.)},%
		ymin=0,ystep=5000,ymax=35000.0,%
		scaled y ticks = false,%
		ymajorgrids = true,
		legend style={at={(0.02,10)}},legend pos=north east]%
		\addplot table [x=no, y=biexciton] {./data/line2.csv}; %
		\addlegendentry {biexciton}%
		\addplot table [x=no, y=exciton] {./data/line2.csv}; %
		\addlegendentry {exciton}%
		\end{axis}
		\node at (-0.9, 2.9) {(b)};
		\end{tikzpicture}
	\end{tabular}
	\caption{(a) Route set to line 2. (b)  is the analyzed data and shows the tendency in the path along line 2.}
	\label{fig:line2}  
\end{figure}


Figure \ref{fig:line2}, 즉 line 2에서는 exciton과 biexciton 모두 감소하는 추세를 보이다가 가장 끝 두점에서는 biexciton은 급격히 증가, exciton은 급격히 감소함을 볼 수 있다.

\begin{figure}[H]
	\begin{tabular}{cc}
		\includegraphics[width=0.3\textwidth]{line3}
		\begin{tikzpicture} [remember picture,overlay]	
		\node[text=white] at (-4, 4) {(a)};
		\end{tikzpicture}
		&
		\begin{tikzpicture}
		\begin{axis} [
		width=0.70\textwidth,%
		height = 4cm,%
		ybar,%
		bar width=5pt,
		title={Line 3},%
		xtick = data,%
		symbolic x coords={0, 1, 2, 3, 4, 5, 6, 7, 8, 9, 10, 11, 12, 13, 14},%
		xlabel= {Viewpoint},%
		ylabel= {Intensity(a.u.)},%
		ymin=0,ystep=5000,ymax=35000.0,%
		scaled y ticks = false,%
		ymajorgrids = true,
		legend style={at={(0.02,10)}},legend pos=north east]%
		\addplot table [x=no, y=biexciton] {./data/line3.csv}; %
		\addlegendentry {biexciton}%
		\addplot table [x=no, y=exciton] {./data/line3.csv}; %
		\addlegendentry {exciton}%
		\end{axis}
		\node at (-0.9, 2.9) {(b)};
		\end{tikzpicture}
	\end{tabular}
	\caption{(a) Route set to line 3. (b)  is the analyzed data and shows the tendency in the path along line 3.}
	\label{fig:line3}  
\end{figure}





Figure \ref{fig:line3}, 즉 line 3에서는 exciton은 감소, biexciton은 증가하는 추세를 보이다가 가장 끝 두 점에서는 biexciton은 급격히 감소, exciton은 급격히 증가함을 볼 수 있다.

세 line에서 exciton, biexciton 각각의 공통되는 경향성이나 규칙은 찾아보기 어렵다. 하지만 중앙에서 중간까지 갈 때는 특정한 경향성을 보이는 듯하다가 가장 바깥, 가장자리에서 그 경향성이 반대되는 모습을 볼 수 있다. 종합적으로 보았을 때는 가장자리로 가면서 감소하는 모습을 보이다가 다시 증가하는 모습이 세 line 모두에서 나타났다.

같은 ND2로 찍은 PL 데이터를 관찰했을 때, 완전한 가장자리를 제외하면 바깥으로 갈수록 biexciton peak의 상대적인 세기가 세짐을 관찰할 수 있었다. 

PL 측정 시 $\rm CsPbBr_3$가 구조상의 deformation이 일어나지 않으므로 비슷한 양의 carrier가 전도띠로 가는 것은 자명하다. 이 carrier들은 각각 exciton이나 biexciton의 형태로 존재하게 되는데, PL에서 biexciton에 의한 peak가 더 우세하게 관찰된 것이라고 해석할 수 있다.


	