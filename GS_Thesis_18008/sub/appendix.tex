%\clearpage  %%% Appendix를 새 페이지에서 시작
\appendix
\renewcommand{\thesection}{\Alph{section}} %%% TOC에 appendix numbering 재설정
\renewcommand{\thesubsection}{\arabic{subsection}}
\renewcommand{\thesubsubsection}{\arabic{subsubsection}}
\titleformat{\section}[hang] {\normalfont\fontsize{21}{21}\selectfont\bfseries}{\Alph{section}.}{1em}{} %%% Appendix section title의 재설정
\titleformat{\subsection}[hang] {\normalfont\fontsize{16}{16}\selectfont\bfseries}{\Alph{section}.\arabic{subsection}.}{1em}{}
\titleformat{\subsubsection}[hang] {\normalfont\fontsize{14}{14}\selectfont}{\Alph{section}.\arabic{subsection}.\arabic{subsubsection}.}{1em}{}
\titleformat{\paragraph}[hang] {\normalfont\fontsize{12}{12}\selectfont\it}{}{1em}{}
\renewcommand{\theequation}{\thesection.\arabic{equation}} %%% Appendix equation numbering 의 재설정
\renewcommand{\thefigure}{\thesection-\arabic{figure}} %%% Appendix figure numbering 의 재설정
\renewcommand{\thetable}{\thesection-\arabic{table}} %%% Appendix table numbering 의 재설정
\setcounter{equation}{0} %%% Appendix equation starting number의 초기화
\setcounter{figure}{0} %%% Appendix figure starting number의 초기화
\setcounter{table}{0} %%% Appendix table starting number의 초기화

\section{부록}

\subsection{Firmware 코드}

\subsubsection{GS-system.ino}
\lstset{basicstyle=\scriptsize, tabsize=4, numbers=left, keywordstyle=\color{blue}, commentstyle=\color{magenta}}
\lstinputlisting[language=c++]{GS-system/GS-system.ino}

\subsubsection{Board.h}
\lstset{basicstyle=\scriptsize, tabsize=4, numbers=left, keywordstyle=\color{blue}, commentstyle=\color{magenta}}
\lstinputlisting[language=c++]{GS-system/Board.h}

\subsubsection{ButtonControl.ino}
\lstset{basicstyle=\scriptsize, tabsize=4, numbers=left, keywordstyle=\color{blue}, commentstyle=\color{magenta}}
\lstinputlisting[language=c++]{GS-system/ButtonControl.ino}

\subsubsection{displayAdafruit.ino}
\lstset{basicstyle=\scriptsize, tabsize=4, numbers=left, keywordstyle=\color{blue}, commentstyle=\color{magenta}}
\lstinputlisting[language=c++]{GS-system/displayAdafruit.ino}

\subsubsection{EEPROM.ino}
\lstset{basicstyle=\scriptsize, tabsize=4, numbers=left, keywordstyle=\color{blue}, commentstyle=\color{magenta}}
\lstinputlisting[language=c++]{GS-system/EEPROM.ino}

\subsubsection{MainControl.ino}
\lstset{basicstyle=\scriptsize, tabsize=4, numbers=left, keywordstyle=\color{blue}, commentstyle=\color{magenta}}
\lstinputlisting[language=c++]{GS-system/MainControl.ino}

\subsubsection{Setstep.ino}
\lstset{basicstyle=\scriptsize, tabsize=4, numbers=left, keywordstyle=\color{blue}, commentstyle=\color{magenta}}
\lstinputlisting[language=c++]{GS-system/Setstep.ino}



\subsection{ASCOM driver 코드}

\lstset{basicstyle=\scriptsize, tabsize=4, numbers=left, keywordstyle=\color{blue}, commentstyle=\color{magenta}}

\begin{lstlisting}[language=c++]
# put code here

\end{lstlisting}