\maketitle  % command to print the title page with above variables
\setcounter{page}{1}
%---------------------------------------------------------------------
%                  영문 초록을 입력하시오
%---------------------------------------------------------------------
\begin{abstracts}     %this creates the heading for the abstract page
	\addcontentsline{toc}{section}{Abstract}  %%% TOC에 표시
	\noindent{
	This study have developed GS-system which is a control system capable to remote control the acrylic bahtinov mask, so it can help automating celestial tube using small astronomical telescopes. GS-system used the previously developed motor focuser, but it equipped the remote control of the Bahtinov mask and various other functions to enhance ease of use. It is used to control the cover for controlling the Bahtinov mask that can be used for certain small astronomical telescopes. Through GS-system, the Bahtinov mask used for astronomical telescopes can be conveniently controlled, and the functions required for astromical observation can also be conveniently used using GUI(graphical user interface) through software compatible with ASCOM drivers.
		key word : Telescope, 
	}
\end{abstracts}

\begin{abstractskor}
	본 연구는 아크릴 바흐티노프마스크 원격제어가 가능한 제어시스템인 GS-system를 개발하여 소형 천체망원경을 이용한 천체관측의 자동화에 도움이 될 수 있도록 하였다. GS-system은 기존에 개발한 모터포커서를 사용하되, 바흐티노프마스크의 원격 제어 및 다른 여러 편의기능을 탑재하여 사용의 편의성을 증대시켰으며, 이를 사용하여 특정 소형 천체망원경에 사용할 수 있는 바흐티노프마스크 제어용 덮개를 제어할 수 있도록 하였다. GS-system을 통해 소형 천체망원경에 사용되는 바흐티노프마스크의 제어를 편리하게 할 수 있으며, 천체 관측에 필요한 기능들 또한 ASCOM 드라이버와 호환되는 소프트웨어들을 통한 GUI (graphical user interface)를 활용해 편리하게 사용할 수 있도록 하였다. 
	
	키워드 : 자동망원경
	
\end{abstractskor}
%----------------------------------------------
%   Table of Contents (자동 작성됨)
%----------------------------------------------
\cleardoublepage
\addcontentsline{toc}{section}{Contents}
\setcounter{secnumdepth}{3} % organisational level that receives a numbers
\setcounter{tocdepth}{3}    % print table of contents for level 3
\baselineskip=2.2em
\tableofcontents


%----------------------------------------------
%     List of Figures/Tables (자동 작성됨)
%----------------------------------------------
\cleardoublepage
\clearpage
\listoffigures	% 그림 목록과 캡션을 출력한다. 만약 논문에 그림이 없다면 이 줄의 맨 앞에 %기호를 넣어서 코멘트 처리한다.

\cleardoublepage
\clearpage
\listoftables  % 표 목록과 캡션을 출력한다. 만약 논문에 표가 없다면 이 줄의 맨 앞에 %기호를 넣어서 코멘트 처리한다.

%%%%%%%%%%%%%%%%%%%%%%%%%%%%%%%%%%%%%%%%%%%%%%%%%%%%%%%%%%%
%%%% Main Document %%%%%%%%%%%%%%%%%%%%%%%%%%%%%%%%%%%%%%%%
%%%%%%%%%%%%%%%%%%%%%%%%%%%%%%%%%%%%%%%%%%%%%%%%%%%%%%%%%%%
\cleardoublepage
\clearpage
\renewcommand{\thepage}{\arabic{page}}
\setcounter{page}{1}



