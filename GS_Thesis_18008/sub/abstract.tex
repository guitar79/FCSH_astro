\maketitle  % command to print the title page with above variables
\setcounter{page}{1}
%---------------------------------------------------------------------
%                  영문 초록을 입력하시오
%---------------------------------------------------------------------
\begin{abstracts}     %this creates the heading for the abstract page
	\addcontentsline{toc}{section}{Abstract}  %%% TOC에 표시
	\noindent{
This study developed GS-system for the automation of small astronomical telescopes. As astronomical telescopes become popular now, the automation process of small astronomical telescopes is drawing attention. Thus, by developing and controlling a light pain cover to control the Bahtinov mask, various convenient functions needed for the automation of the astronomical tube were provided. GS-system is divided into drive part and control part. The drive part consists of the control of the cover of the passage described earlier and hardware for convenience functions such as thermic rays control, power control, and motor control. The control part also improves the existing motor focuser and ASCOM compatible driver to allow control of newly controlled parts using the existing ASCOM-enabled software. GS-touch is believed to be of great help to the astronomical observation of small telescopes in the future.
	
		key word : Telescope, Automization, Teensy
	}
\end{abstracts}

\begin{abstractskor}
본 연구에서는 소형 천체망원경의 자동화를 위한 GS-system을 개발하였다. 현재 천체망원경이 대중화됨에 따라 소형 천체망원경의 자동화 과정이 주목을 받고 있다. 이에 바흐티노프 마스크를 제어할 수 있는 경통의 덮개를 개발하여 이를 제어함으로써 천체관측의 자동화에 필요한 여러 가지 편의기능들을 제공할 수 있도록 하였다. GS-system은 구동부와 제어부로 나뉜다.  구동부에서는 앞서 설명한 경통의 덮개 및 열선 제어, 전원 제어, 초점 제어 등의 편의기능들을 위한 하드웨어들로 구성되어 있다. 제어부는 기존의 모터포커서 및 ASCOM 호환 드라이버를 개선하여 구동부를 제어할 수 있도록 하였다. GS-touch는 앞으로의 소형 천체망원경의 천체관측에 많은 도움이 될 수 있을 것으로 사료된다.

키워드 : 자동망원경, 자동화, Teensy
	
\end{abstractskor}
%----------------------------------------------
%   Table of Contents (자동 작성됨)
%----------------------------------------------
\cleardoublepage
\addcontentsline{toc}{section}{Contents}
\setcounter{secnumdepth}{3} % organisational level that receives a numbers
\setcounter{tocdepth}{3}    % print table of contents for level 3
\baselineskip=2.2em
\tableofcontents


%----------------------------------------------
%     List of Figures/Tables (자동 작성됨)
%----------------------------------------------
\cleardoublepage
\clearpage
\listoffigures	% 그림 목록과 캡션을 출력한다. 만약 논문에 그림이 없다면 이 줄의 맨 앞에 %기호를 넣어서 코멘트 처리한다.

\cleardoublepage
\clearpage
\listoftables  % 표 목록과 캡션을 출력한다. 만약 논문에 표가 없다면 이 줄의 맨 앞에 %기호를 넣어서 코멘트 처리한다.

%%%%%%%%%%%%%%%%%%%%%%%%%%%%%%%%%%%%%%%%%%%%%%%%%%%%%%%%%%%
%%%% Main Document %%%%%%%%%%%%%%%%%%%%%%%%%%%%%%%%%%%%%%%%
%%%%%%%%%%%%%%%%%%%%%%%%%%%%%%%%%%%%%%%%%%%%%%%%%%%%%%%%%%%
\cleardoublepage
\clearpage
\renewcommand{\thepage}{\arabic{page}}
\setcounter{page}{1}



