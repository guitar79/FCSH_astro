%-----------------------------------------------------
% Conclusion
%-----------------------------------------------------
\section{결론}
본 연구에서는 소형 천체망원경의 자동화를 위한 GS-system을 개발하였다. 
소형 천체망원경을 이용한 천체관측의 자동화를 하기 위해 필요한 여러 시스템들을 기존에 개발된 모터포커서를 이용하여 실제 관측에서 사용될 수 있는 것을 목적으로 하였으며, 이를 위한 바흐티노프 마스크 제어용 망원경 덮개의 개발 및 시험 동작을 시행하였으며, 다른 여러 편의기능 또한 시험동작을 완료하였다. 본 제어시스템을 개발하기 위해 사용한 틀은 EasyEDA, Fusion 360, Arduino IDE, Visual Studio 2017, ASCOM platform 6등을 포함하며 이들의 효율적인 사용을 위한 여러 라이브러리가 사용되었다.

본 연구의 결과물은 같이 요약할 수 있다. 

\begin{enumerate}
	
	\item 천체망원경 자동화를 위한 전원 스위치, 모터 포커서, 망원경 덮개, 이슬 제거용 열선 등을 포함하는 구동부를 제작하였다. 
	
		\item 3D프린터와 망원경의 경통 구조를 적절히 활용하여 바흐티노프 마스크의 제어가 가능한 덮개를 개발하였다.
	
	
	\item 천체망원경 자동화를 위한 전원 스위치, 모터 포커서, 망원경 덮개, 이슬 제거용 열선 등을 제어할 수 있는 회로를 구성하였다.
	
	\item 천체망원경 자동화를 위한 전원 스위치, 모터 포커서, 망원경 덮개, 이슬 제거용 열선 등을 제어할 수 있는 ASCOM 호환 소프트웨어를 개발하였다.
	
	
\end{enumerate}

먼저, 3d프린터와 망원경의 경통 구조를 적절히 활용하여 바흐티노프 마스크의 제어가 가능한 덮개를 개발하였다. 덮개의 크기 때문에 저반적인 3d프린터로 출력하기 힘들어 네 부분으로 나누어 출력하였으며, 경첩을 이용하여 마스크를 제어하기 때문에 경첩이 들어갈 부분을 만들어주었다. 경첩은 같은 방향에 부착된 서보모터를 이용하여 제어되며, 서보모터는 항상 같은 각도로 움직일 수 있기 때문에 경첩과 마스크를 동작시킬때의 안정성이 매우 뛰어나다.

또한, 천체망원경의 원격 제어 및 덮개의 원활한 동작을 위해 모터포커서를 개선하였다. 개선한 부분은 크게 네 가지로. 이슬 제거용 열선의 제어, EEPROM 적용, 덮개의 제어를 위한 서보모터 제어, 전원 제어의 편리함을 더하는 릴레이스위치의 제어로 나놀 수 있다. 또한 새로 생긴 기능들을 적절히 활용할 수 있도록 포커서 컨트롤 소프트웨어의 기능들 또한 개선하였으며, 이들의 기능중 일부를 오프라인에서 Menu System을 활용하여 제어할 수 있도록 하였다.

마지막으로 기존에 모터의 초점 제어용으로 개발되었던 GS-touch의 ASCOM 호환 드라이버를 개선하여 앞서 만들었던 기능들을 PC를 통해 전달할 수 있도록 하였다. 모든 기능들은 드라이버 상의 GUI에 나타나도록 하여 제어가 가능하도록 하였다.


현재 망원경을 사용하기 위해서는 미리 덮개를 열고 있어야 한다. 하지만 덮개를 열고 있는 상태에서 먼지가 경통의 속으로 들어가게 되는 단점이 존재하게 되어 이는 원격 천체망원경의 제어에 큰 어려움이 된다. 하지만 이번 연구를 통해서 덮개를 미리 열고 있어야 하는 단점을 없애 먼지가 최대한 안 들어오도록 할 수 있게 되어 소형 천체망원경을 활용하여 관측할 때 직접 망원경을 확인해야 하는 수고를 덜었으며, 덮개로 바흐티노프마스크를 제어할 수 있게 되어 포커서 컨트롤러를 만들어 초점을 맞출 때 C-curve를 그릴 필요 없이 초점을 맞출 수 있게 되었다. 또한 열선을 설치하여 렌즈에 이슬이 맺히는 것을 방지하여 좀 더 좋은 별 상을 얻을 수 있게 될 것으로 예상된다. 위 결과들을 종합해보았을 때, 직접 개발한 시스템인 GS-system은 약간의 개선을 통해 소형 천체망원경의 자동화된 관측에 사용될 수 있을 것으로 예상된다.


본 연구에서 아쉬운 부분은 PCB의 회로를 설계할 때 RX,TX (Teensy 3.2 상에서 0번, 1번 핀)이 SerialPort의 활성화를 위한 핀임을 인지하지 못하여 다른 편의기능들은 모두 제어가 가능하지만 모터의 초점 포커싱 기능을 사용하지 못한다는 점이다. 이 점은 앞으로의 개선사항과 더하여 Teensy 3.2 상의 핀 배열을 수정하여 해결할 수 있을 것으로 사료된다.